\documentclass[11pt]{article}
%common citations: yu1989fixed,
%Page Format control
%-------------------------------------
%formatting post: 
%https://texblog.org/2012/03/01/latex-page-line-and-font-settings/
\usepackage[top=1in, bottom=1in, left=1.25in, right=1.25in]{geometry}
\usepackage{setspace}
\usepackage{amsmath}
\usepackage{hyperref}
\usepackage{graphicx}
\usepackage{subfigure}
\onehalfspacing
\usepackage{fancyhdr}
\usepackage{multirow}
\usepackage{ytableau}
\usepackage{makecell}
\usepackage{graphicx}




%Packages
\usepackage{amssymb,amsmath, amsthm, graphicx, color,subfigure, cancel,commath,ifthen,natbib, bm,alltt, graphicx,float}
%external definitions
%-------------------------------------
%\input{HeaderBH}
%multiline comment
\newcommand\commentout[1]{}
\graphicspath{ {./images/} }
%-------------------------------------

%-------
%controllig equation numbering
%https://tex.stackexchange.com/questions/42726/align-but-show-one-equation-%number-at-the-end
\newcommand\numberthis{\addtocounter{equation}{1}\tag{\theequation}}
%or use align with \nonumber
%-------

\usepackage{xr} %These two lines allow cross  reference from the supplementary file. 
\externaldocument{CombDiscreteTests_Suppl}
\newcommand{\nc}{\newcommand}
%\newcommand{\E}{\mathrm{E}}
\newcommand{\E}{\mathbb{E}}
\newcommand{\Var}{\mathrm{Var}}
\newcommand{\SD}{\mathrm{SD}}
\newcommand{\SNR}{\text{SNR}}
\newcommand{\Cov}{\mathrm{Cov}}
\newcommand{\Cor}{\mathrm{Cor}}
\renewcommand{\P}{\mathbb{P}}
\newcommand{\R}{\mathbb{R}}
\newcommand{\SIGMA}{\bm{\Sigma}}
\newcommand{\BETA}{\bm{\beta}}
\newcommand{\GAMMA}{\bm{\gamma}}


%environments
%=======================
\newtheorem{defi}{Definition}
\newtheorem{thm}{Theorem}
\newtheorem{obs}{Observation}
\newtheorem{cor}{Corollary}
\newtheorem{lem}{Lemma}
\newtheorem{ass}{Assumption}
%=======================


\begin{document}
\thispagestyle{fancy}

%===================================================
\title{\bf Prediction Accuracy Over Sample Sizes and Added-on Non-basic Predictors}
%\author{...
%\and
%...
%}
\date{}
\maketitle
% 
% \begin{abstract}
%...
%
%
%\bigskip 
%
%\noindent \textbf{KEYWORDS:} 
% \end{abstract}
%%===================================================
%%END: frontmatter
%
%%BEGIN: \section{Introduction}\label{sec:Intro}
%%===================================================
%\newpage
%

%=======================
\section{Aim}

This report aims to use calculations or simulations to demonstrate the relationship between sample size and newly added predictors. We believe that the more variables are added, the smaller the efficient sample size, that is, the prediction accuracy tends to be more stable and does not increase significantly when more predictors are added and using a smaller sample size.

\section{Result}


Based on the PMSE improvement document and SampleSizeAnalysisEPPICWu. The data is generated with the covariate matrix $\boldsymbol{\Sigma}$ from \cite{baker2008chronicpain}. The "Basic" model contains $p=3$ demographic predictors.
The "non-basic" predictors are added sequentially into the model being evaluated by the rPMSEp. The calculated rPMSEp is shown as \ref{table:1}. The efficient sample size with $\alpha = 0.1$ (i.e. reaches 90\% of the largest pPMSEr at n = $\infty$)is calculated as a reference in  \ref{table:2}.

\begin{table}[h!]
\centering
\resizebox{\linewidth}{!}{ 
\begin{tabular}{||c c c c c c c c c c||}
 \hline
 \makecell[c]{Sample\\ Size} & \makecell[c]{Basic \\ Predictors} &Comorbidities & Pain Locations & Medications & \makecell[c]{Physical \\ Functioning} & \makecell[c]{Depressive \\Symptoms} & \makecell[c]{Life\\ Satisfaction} & \makecell[c]{LOC\\-Chance} & \makecell[c]{LOC\\-Powerful} \\ [0.5ex] 
 \hline\hline
 60 & 0.4033 & 0.2884 & 0.2512 & 0.2559 & 0.2041 & 0.1919 & 0.1806 & 0.0549 & 0.0327\\ 
 90 & 0.4419 & 0.3300 & 0.2903 & 0.2898 & 0.2349	& 0.2174 & 0.2004 & 0.0704 & 0.0408\\
 120 & 0.4586 & 0.3481 & 0.3072 & 0.3045 & 0.2482	 & 0.2285	& 0.2090 & 0.0771 & 0.0443\\
 150 & 0.4680 & 0.3582 & 0.3167 & 0.3127 & 0.2557 & 0.2347	& 0.2138 & 0.0809 & 0.0463\\
 180 & 0.4739 & 0.3646 & 0.3227 & 0.3180 & 0.2605	 & 0.2387	& 0.2169 & 0.0833 & 0.0476\\ 
 210 & 0.4781 & 0.3691 & 0.3269 & 0.3216 & 0.2638	 & 0.2414	& 0.2190 & 0.0850 & 0.0484\\ 
 240 & 0.4811 & 0.3724 & 0.3300 & 0.3243 & 0.2662	 & 0.2434	& 0.2206 & 0.0862 & 0.0491\\ 
 270 & 0.4834 & 0.3749 & 0.3323 & 0.3264 & 0.2681	 & 0.2449	& 0.2218 & 0.0872 & 0.0496\\ 
 300 & 0.4853 & 0.3769 & 0.3342 & 0.3280 & 0.2695	 & 0.2462	& 0.2227 & 0.0879 & 0.0500\\ 
 330 & 0.4868 & 0.3785 & 0.3357 & 0.3293 & 0.2707	 & 0.2472	& 0.2235 & 0.0885 & 0.0503\\ 
 360 & 0.4880 & 0.3798 & 0.3370 & 0.3304 & 0.2717	 & 0.2480	& 0.2241 & 0.0890 & 0.0505\\ [1ex] 
 \hline
\end{tabular}}
\caption{rPMSEp Sequentially added predictors over the “basic” 3 predictors}
\label{table:1}
\end{table}

add calculated PMSE Table


\begin{table}[h!]
\centering
\resizebox{\linewidth}{!}{ 
\begin{tabular}{||c c c c c c c c c||}
 \hline
\makecell[c]{Basic \\ Predictors} &Comorbidities & \makecell[c]{Pain\\ Locations} & Medications & \makecell[c]{Physical \\ Functioning} & \makecell[c]{Depressive \\Symptoms} & \makecell[c]{Life\\ Satisfaction} & \makecell[c]{LOC\\-Chance} & \makecell[c]{LOC\\-Powerful} \\ [0.5ex] 
 \hline\hline
 103.6487 & 137.1353 & 143.9351 & 129.5519 & 141.2487 & 129.9053 & 113.9951 & 206.2313 & 191.7463\\[1ex]
 \hline
\end{tabular}}
\caption{Efficient sample size $n^*$}
\label{table:2}
\end{table}

\begin{itemize}

\item The value in \ref{table:1} shows the proportion of variation explained by different combinations of the predictor variables
The complete model includes an additional predictor, $LOC-internal$, which is not listed in the table. This is because the rPMSEp was compared to the full model, which resulted in a rPMSEp value of 1.

\item Once the sample size reached the efficient sample size of $n^*$, the rPMSEp was expected to remain stable as the sample size continued to increase. This result confirms the formula of efficient sample size $n^*$.

\item When insignificant predictors are incorporated into the model, the efficient sample size $n^*$ will not decrease but rather increase. Including these predictors does not yield additional information but instead introduces random error, which implies that achieving the same prediction accuracy would require a larger sample size.

\
\end{itemize}


%----------------------------------
\bibliographystyle{ims}
\bibliography{PredictionSampleSize}
%----------------------------------




\end{document}

