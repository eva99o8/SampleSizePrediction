\begin{abstract}
    
%This thesis aims to investigate the relationship between prediction accuracy and extra added predictors considering sample size. The goal is to analyze the impact of various factors on the prediction mean square error ($PMSE$) and develop guidelines for study designs that can improve the accuracy and reliability of research results. Specifically, we investigate the relationship between sample size, number of predictors, and effect sizes on the $PMSE$, and also summarize the degree of influence of added predictors on the model's prediction accuracy. To quantify this influence, we propose the use of the reduced prediction mean square error percentage ($rPMSEp$), which corresponds to the efficient sample size required to achieve a certain prediction accuracy. Our findings are supported by calculations based on data from the pain study. The results of this research can inform future study designs and contribute to the development of more accurate prediction models.

This work explores the relationship between prediction accuracy, the impact of additional predictors, and sample size in the context of multiple linear regression models. The objective is to facilitate sample size calculations for study designs that directly target predictive power (i.e., prediction accuracy) in various applicational studies. To achieve this goal, we analyze the functional relationship between prediction mean square error (PMSE) and factors such as the number, effect sizes, and correlations among predictors, as well as sample size.

Building on this analysis, we introduce a metric referred to as the percentage of PMSE reduction (pPMSEr) to quantify the improvement in prediction accuracy when sample size is increased and/or new important predictors are added to a model. Given a set of predictors, we can compute an efficient sample size, defined as the smallest sample size that achieves, for example, 90\% of the prediction accuracy ever achievable at an infinite sample size. Beyond this efficient sample size, increasing the sample size does not significantly improve prediction accuracy unless more important predictors are incorporated into the model.

We validate these calculations through computations and simulations based on a pain study, demonstrating a practical application and interpretation of the proposed measures in planning prediction-related studies.



\end{abstract}